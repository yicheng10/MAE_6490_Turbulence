% =============== package to use and general setting===================
\documentclass[11pt,letterpaper]{article}
\usepackage[backend=biber]{biblatex}
\usepackage[margin=0.5 in]{geometry}
\usepackage{setspace}

\onehalfspacing{}
\setlength\parindent{0pt}
\setlength\parskip{10pt}

\usepackage{url, amsfonts, amssymb, amsmath, color,
            enumitem, float, graphicx, subfigure, float, tikz, pgfplots}

\usepackage[skins]{tcolorbox}
\pgfplotsset{compat=1.18}
\mathchardef\mhyphen="2D

% ========= packages for insert programming code ======================
\usepackage{listings}
\usepackage{xcolor}

\definecolor{codegreen}{rgb}{0,0.6,0}
\definecolor{codegray}{rgb}{0.5,0.5,0.5}
\definecolor{codepurple}{rgb}{0.58,0,0.82}
\definecolor{backcolour}{rgb}{0.95,0.95,0.92}

\lstdefinestyle{mystyle}{
    backgroundcolor=\color{backcolour},
    commentstyle=\color{codegreen},
    keywordstyle=\color{magenta},
    numberstyle=\tiny\color{codegray},
    stringstyle=\color{codepurple},
    basicstyle=\ttfamily\footnotesize,
    breakatwhitespace=false,
    breaklines=true,
    captionpos=b,
    keepspaces=true,
    numbers=left,
    numbersep=5pt,
    showspaces=false,
    showstringspaces=false,
    showtabs=false,
    tabsize=2
}

\lstset{style=mystyle}
% ===================== title =========================================


\begin{document}
\begin{center} \Large
    \textbf{Assignment 2}
\end{center}
\begin{center}A02394870 \\ Yi-Cheng, Chen \end{center}

% ===================== Question 1 ====================================
\begin{enumerate}
    \item \textbf{Velocity fields and outer units}
          \begin{enumerate}
              \item The plots as following:
                    \begin{figure}[H]
                        \centering
                        \subfigure[name1]{
                            \label{Fig.sub.1}
                            \includegraphics[width=0.8\textwidth]{1a_1.pdf}}
                        \subfigure[name2]{
                            \label{Fig.sub.2}
                            \includegraphics[width=0.8\textwidth]{1a_2.pdf}}
                        \subfigure[name2]{
                            \label{Fig.sub.3}
                            \includegraphics[width=0.8\textwidth]{1a_3.pdf}}
                        \caption{$u_1^{\prime}$ and $u_2^{\prime}$ contour}
                        \label{Fig.main}
                    \end{figure}
              \item The free-stream velocity is the velocity from most outer layer in y direction.
                    To avoid the measurement error, I took the last 10 layar in y direction and average them both spatial and temporal.
                    For boundary-layer thickness $\delta$, just find where $\overline{u}(\delta) = 0.99 U_{\infty}$.
                    For displacement thickness $\delta^*$ equation as follows
                    \begin{equation}
                        \delta^* = \int_{0}^{\infty} (1 - \frac{u}{u_\infty})dy
                    \end{equation}
                    \begin{equation}
                        \theta = \int_{0}^{\infty} \frac{u}{u_\infty} (1 - \frac{u}{u_\infty})dy
                    \end{equation}
                    \begin{equation}
                        TI = \frac{\sqrt[]{ \langle \overline{u^\prime} \rangle^2 + \langle \overline{v^\prime} \rangle}}{U_\infty}
                    \end{equation}
              \item The plot as follows
                    \begin{figure}[H]
                        \centering
                        \includegraphics[width=0.7\textwidth]{1c.pdf}
                        \caption{title}
                        \label{1c}
                    \end{figure}
              \item this is 1d
              \item this is 1e
                    \begin{figure}[H]
                        \centering
                        \includegraphics[width=0.7\textwidth]{1e.pdf}
                        \caption{title}
                        \label{1e}
                    \end{figure}
          \end{enumerate}
    \item \textbf{Friction velocity}
    \item \textbf{Wall-normal profiless}
    \item \textbf{Uniform momentum zones (UMZs)}
    \item \textbf{Quadrant analysis}
\end{enumerate}




% ===================== Python Code ===================================
\section*{\centering{Appendix A: Python code for Question 1 }}
\begin{lstlisting}[language=Python, caption=Qusetion 1 Python code]
import numpy as np
import scipy
import matplotlib.pyplot as plt
import matplotlib as mpl

# global setting
L = 1
c = 1

Peclet = 20
n = 4
plt.rcParams['figure.figsize'] = (8, 8)
mpl.rcParams['text.usetex'] = True


def get_analytic_sol(Pe, n):
    # ================ get exact solutio eq(30) =======================
    h = L/n
    unknow = n - 1
    a = (c*L) / Pe
    u_exact = np.zeros((unknow))
    x = [(i+1) * h for i in range(unknow)]

    for i in range(0, n-1):
        eta = x[i] /L
        # print("x = ", x[i])
        u_exact[i] = (L/c) * (eta - (np.exp(Pe * (eta - 1)) - np.exp(- Pe)) / (1 - np.exp(- Pe)))
    return u_exact

def get_discrete_analy_sol(Pe, n):
    # ================ get discrete solution eq(33) ===================
    h = L/n
    a = (c*L) / Pe
    unknow = n - 1

    u_exact = np.zeros((unknow))
    x = [(i+1) * h for i in range(unknow)]
    Pe_g = c * h / a
    gamma = (2 + Pe_g) / (2 - Pe_g)
    print("n = ", n, ", Pe_g = ", Pe_g)

    for i in range(0, n-1):
        u_exact[i] = L/c * ((x[i] / L) - (1 - gamma**i) / (1 - gamma**n))
    return u_exact, Pe_g

def get_H(Pe, n):
    # ================ get discrete solution eq(32) ===================
    unknow = n - 1
    a = (c*L) / Pe
    h = L/n
    # ================ create Diffusion_matrix ========================
    u_spd1 = np.stack((np.ones(unknow), -2 * np.ones(unknow), np.ones(unknow)))
    diags = np.array([1, 0, -1])
    A = (-1/Pe) * scipy.sparse.spdiags(u_spd1, diags, unknow, unknow).toarray() / h**2

    # ================ create Advection_matrix ========================
    u_spd2 = np.stack((np.ones(unknow), 0 * np.ones(unknow), -1* np.ones(unknow)))
    diags = np.array([1, 0, -1])
    C = scipy.sparse.spdiags(u_spd2, diags, unknow, unknow).toarray() / (2*h)

    H = A + C # Advection-Diffusion operator
    return H


# ============ Plot solution of u =====================================
Peclet = [1, 10, 100, 1000]
n = [4, 8, 16, 32, 64]
markers = ['s', 'v', '.', 'x', '+' ]

ax = plt.subplot()
linewide = 1

for i in range(4): # 4 Peclet
    ax = plt.subplot()
    for j in range(5): # 5 n
        # x position


        # eq32

        pe = float(Peclet[i])
        H = get_H(Peclet[i], n[j])
        B = np.ones(n[j] - 1)
        u_solution = np.linalg.solve(H, B)
        xb1 = [i * L/n[j] for i in range(n[j] + 1)]
        y  = np.concatenate((np.array([0]), u_solution, np.array([0])))
        if n[j] > 60:
            ax.plot(xb1, y, ls='None', label="eq (31), n = " + str(n[j]), c='r', marker=markers[j] , lw = linewide)
        else:
            ax.plot(xb1, y, ls='-.', label="eq (31), n = " + str(n[j]), c='r', marker=markers[j] , lw = linewide)

    for j in range(5): # 5 n
        # eq31
        discr_sol, peg = get_discrete_analy_sol(Peclet[i], n[j])
        xb2 = [i * L/n[j] for i in range(n[j] + 1)]
        y_discr  = np.concatenate((np.array([0]), discr_sol, np.array([0])))
        if n[j] > 60:
            ax.plot(xb2, y_discr, ls='None' , label="eq (32), n = " + str(n[j]) + r", $Pe_g = $" + str(peg), c='tab:blue', marker=markers[j], lw = linewide)
        else:
            ax.plot(xb2, y_discr, ls='-' , label="eq (32), n = " + str(n[j]) + r", $Pe_g = $" + str(peg), c='tab:blue', marker=markers[j], lw = linewide)

    # eq31
    analy_sol = get_analytic_sol(Peclet[i], 1024)
    xb = [i * L/1024 for i in range(1024 + 1)]
    y_analy = np.concatenate((np.array([0]), analy_sol[:], np.array([0])))
    ax.plot(xb, y_analy , label="eq (30)", ls='-', c='black', lw = linewide)

    ax.set_xlabel("L")
    ax.set_ylabel("u")
    if i > 1:
        ax.set_ylim(0, 1.5)
    ax.legend(loc='upper left', fontsize=15)
    plt.savefig('Pe_' + str(Peclet[i]) +'.png', bbox_inches="tight")
    plt.show()


# ============ Plot || u - u~ || / ||u~|| =============================

n_list = [2 ** i for i in range(1, 16)]

error_pe_1 = np.zeros(len(n_list))
error_pe_10 = np.zeros(len(n_list))
error_pe_100 = np.zeros(len(n_list))
error_pe_1000 = np.zeros(len(n_list))

Peclet = 1
for i in range(len(n_list)):
    analy_sol = get_analytic_sol(Peclet, n_list[i])
    H = get_H(Peclet, n_list[i])
    B = np.ones(n_list[i] - 1)
    u_solution = np.linalg.solve(H, B)
    error_pe_1[i] = np.linalg.norm((u_solution - analy_sol), ord=np.inf) / np.linalg.norm(analy_sol, ord=np.inf)

Peclet = 10
for i in range(len(n_list)):
    analy_sol = get_analytic_sol(Peclet, n_list[i])
    H = get_H(Peclet, n_list[i])
    B = np.ones(n_list[i] - 1)
    u_solution = np.linalg.solve(H, B)
    error_pe_10[i] = np.linalg.norm((u_solution - analy_sol), ord=np.inf) / np.linalg.norm(analy_sol, ord=np.inf)

Peclet = 100
for i in range(len(n_list)):
    analy_sol = get_analytic_sol(Peclet, n_list[i])
    H = get_H(Peclet, n_list[i])
    B = np.ones(n_list[i] - 1)
    u_solution = np.linalg.solve(H, B)
    error_pe_100[i] = np.linalg.norm((u_solution - analy_sol), ord=np.inf) / np.linalg.norm(analy_sol, ord=np.inf)

Peclet = 1000
for i in range(len(n_list)):
    analy_sol = get_analytic_sol(Peclet, n_list[i])
    H = get_H(Peclet, n_list[i])
    B = np.ones(n_list[i] - 1)
    u_solution = np.linalg.solve(H, B)
    error_pe_1000[i] = np.linalg.norm((u_solution - analy_sol), ord=np.inf) / np.linalg.norm(analy_sol, ord=np.inf)

ax = plt.subplot()
ax.plot(n_list , error_pe_1,       label=r"$Pe = 1$",      marker='x', lw = linewide)
ax.plot(n_list , error_pe_10,      label=r"$Pe = 10$",     marker='.', lw = linewide)
ax.plot(n_list , error_pe_100,     label=r"$Pe = 100$",    marker='s', lw = linewide)
ax.plot(n_list , error_pe_1000,    label=r"$Pe = 1000$",   marker='v', lw = linewide)
ax.plot([2, 25000], [10**(-8), 10**(-8)],  label=r"error = $10^{-8}$", lw = linewide, c='black', ls='-.')

# ========================== trendline ================================

def curve(x, a, b):
    return a * x ** b

curve_y = [curve(n_list[i], 2, -2) for i in range(0, len(n_list))]
ax.plot(n_list, curve_y, ls=":", label="trendline")

# =====================================================================

ax.set_yscale('log')
ax.set_xscale('log')

ax.set_ylabel(r"$||\underline{u} - \underline{\tilde{u}} ||_\infty / ||\underline{\tilde{u}}||  $")
ax.set_xlabel("n")
ax.legend(fontsize=15)
plt.savefig('Pe_error.png', bbox_inches="tight")
plt.show()
\end{lstlisting}

% ========== Appendix B ================================================
\newpage
\section*{\centering{Appendix B: Python code for Question 2 }}
\begin{lstlisting}[language=Python, caption=Question 2 Python code]
import numpy as np
import scipy as sp
import matplotlib.pyplot as plt

# ==================== define zwgll ===================================
def zwgll(p):
    n = p + 1

    z = np.zeros(n)
    w = np.zeros(n)

    z[0] = -1
    z[-1] = 1

    if p > 1:
        if p == 2:
            z[1] = 0
        else:
            M = np.zeros((p - 1, p - 1))
            for i in range(p - 2):
                M[i, i + 1] = (1/2) * np.sqrt(((i+1) * (i + 3)) / (((i+1) + 1/2) * ((i+1)+ 3/2)))
                M[i + 1, i] = M[i, i + 1]

            D,V = np.linalg.eig(M)
            z[1:p] = np.sort(D)

    w[0] = 2 / (p * n)
    w[-1] = w[0]

    for i in range(1, p):
        x = z[i]
        z0 = 1
        z1 = x

        for j in range(0, p - 1):
            z2 = x * z1 * (2 * (j + 1) + 1) / ((j + 1)+ 1) - z0 * (j + 1) / ((j + 1) + 1)
            z0 = z1
            z1 = z2

        w[i] = 2 / (p * n * z2**2)
    return z, w

# ==================== define deriv_mat ===============================
def deriv_mat(x):
    ni = len(x)
    a = np.ones(ni)

    for i in range(ni):
        for j in range(i):
            a[i] *= (x[i] - x[j])
        for j in range(i + 1, ni):
            a[i] *= (x[i] - x[j])

    a = 1 / a  # These are the alpha_i's

    d = np.zeros((ni, ni))
    for j in range(ni):
        for i in range(ni):
            d[i, j] = x[i] - x[j]
        d[j, j] = 1

    d = 1 / d
    for i in range(ni):
        d[i, i] = 0
        d[i, i] = np.sum(d[i, :])

    for j in range(ni):
        for i in range(ni):
            if i != j:
                d[i, j] = a[j] / (a[i] * (x[i] - x[j]))
    return d

# ==================== define semhat ==================================
def semhat(N):
    z, w = zwgll(N);

    Bh = np.diag(w)
    Dh = deriv_mat(z)
    Ah = np.dot(np.dot(Dh.T, Bh), Dh)
    Ch = np.dot(Bh, Dh)
    return Ah, Bh, Ch, Dh, z, w

# =====================================================================
# Over-sample for error checking
N = 256
M = 400
L = np.pi

d = 1
d2 = d * d

Ah, Bh, Ch, Dh, z, w = semhat(M)

Ab = (2 / L) * Ah
Bb = (L / 2) * Bh
Ib = np.eye(M + 1)
R = Ib[1:-1, :]
A = R @ Ab @ R.T
B = R @ Bb @ R.T
H = d2 * A + B

z = L * (1 + z) / 2
w = L * w / 2
xb = z
fb = 1 + 0 * xb # a all 1 array
us = R.T @ np.linalg.solve(H, R @ Bb @ fb) # solution of Legendre matrix Hx = R @ Bb @ fb

# Initialize Fourier solution on GLL pts
uf =  np.zeros(len(xb))
em =  np.zeros(int(N/2))
ef2 = np.zeros(int(N/2))
nN =  np.zeros(int(N/2))

for k in range(1, N + 1, 2):
    k1 = k + 1
    sk = np.sin(k * xb)
    fk = 2 / (k * np.pi)
    uk = 2 * fk / (1 + d2 * k * k)
    uf = uf + uk * sk # solution of fourier
    ef = uf - us # error of fourier
    em[int(k1/2)-1]  = np.max(np.abs(ef))
    ef2[int(k1/2)-1] = np.sqrt(ef @ Bb @ ef / np.pi)
    nN[int(k1/2)-1]  = k
    if k == 1:
        uf1 = uf
    if k == 3:
        uf3 = uf
    if k == 5:
        uf5 = uf

fig, ax = plt.subplots()
ax.plot(xb, us, 'k-', linewidth=1.4, label = "Legendre")
ax.plot(xb, uf1, 'r--', label = "Fourier k=1")
ax.plot(xb, uf3, 'g--', label = "Fourier k=3")
ax.plot(xb, uf5, 'b--', label = "Fourier k=5")
ax.set_xlabel("x")
ax.set_ylabel("u(x)")
ax.legend()
plt.savefig('2_1.pdf', bbox_inches="tight")
plt.show()

fig, ax = plt.subplots()
ax.plot(nN, ef2,          'b-',  linewidth=1.4, label = "Legendre")
ax.plot(nN, 0.3*nN**(-2), 'r--', label = r'$L_2$ error')
ax.plot(nN, em,           'g--', label = r'$L_\infty$ error')

ax.set_yscale('log')
ax.set_xscale('log')

ax.set_xlabel('N')
ax.set_ylabel(r'Error: $L_2 \ and \ L_\infty$')
ax.legend()
plt.savefig('2_2.pdf', bbox_inches="tight")
plt.show()

    \end{lstlisting}

% ========== Appendix c ================================================
\newpage
\section*{\centering{Appendix C: Python code for Question 2 (b)}}
\begin{lstlisting}[language=Python, caption=Question 2 (b) Python code]
# %%
import numpy as np
import scipy as sp
import matplotlib.pyplot as plt

# ==================== define zwgll ===================================
def zwgll(p):
    n = p + 1

    z = np.zeros(n)
    w = np.zeros(n)

    z[0] = -1
    z[-1] = 1

    if p > 1:
        if p == 2:
            z[1] = 0
        else:
            M = np.zeros((p - 1, p - 1))
            for i in range(p - 2):
                M[i, i + 1] = (1/2) * np.sqrt(((i+1) * (i + 3)) / (((i+1) + 1/2) * ((i+1)+ 3/2)))
                M[i + 1, i] = M[i, i + 1]

            D,V = np.linalg.eig(M)
            z[1:p] = np.sort(D)

    w[0] = 2 / (p * n)
    w[-1] = w[0]

    for i in range(1, p):
        x = z[i]
        z0 = 1
        z1 = x

        for j in range(0, p - 1):
            z2 = x * z1 * (2 * (j + 1) + 1) / ((j + 1)+ 1) - z0 * (j + 1) / ((j + 1) + 1)
            z0 = z1
            z1 = z2

        w[i] = 2 / (p * n * z2**2)
    return z, w


# ==================== define deriv_mat ===============================
def deriv_mat(x):
    ni = len(x)
    a = np.ones(ni)

    for i in range(ni):
        for j in range(i):
            a[i] *= (x[i] - x[j])
        for j in range(i + 1, ni):
            a[i] *= (x[i] - x[j])

    a = 1 / a  # These are the alpha_i's

    d = np.zeros((ni, ni))
    for j in range(ni):
        for i in range(ni):
            d[i, j] = x[i] - x[j]
        d[j, j] = 1

    d = 1 / d
    for i in range(ni):
        d[i, i] = 0
        d[i, i] = np.sum(d[i, :])

    for j in range(ni):
        for i in range(ni):
            if i != j:
                d[i, j] = a[j] / (a[i] * (x[i] - x[j]))
    return d

# ==================== define semhat ==================================
def semhat(N):
    z, w = zwgll(N);

    Bh = np.diag(w)
    Dh = deriv_mat(z)
    Ah = np.dot(np.dot(Dh.T, Bh), Dh)
    Ch = np.dot(Bh, Dh)
    return Ah, Bh, Ch, Dh, z, w

# =====================================================================
# Over-sample for error checking
N = 256
M = 400
L = np.pi

d = 1
d2 = d * d

Ah, Bh, Ch, Dh, z, w = semhat(M)


# %%
Ab = (2 / L) * Ah
Bb = (L / 2) * Bh
Ib = np.eye(M + 1)
R = Ib[1:-1, :]
A = R @ Ab @ R.T
B = R @ Bb @ R.T
H = d2 * A + B
RT = R.T

# Scale points and weights to [-L, L]
# z = L * (1 + z) / 2
z = L * z / 2
w = L * w / 2

# Set RHS for Legendre spectral soln
xb = z
fb = np.cos(xb)

# solution of Legendre matrix Hx = R @ Bb @ fb
us = R.T @ np.linalg.solve(H, R @ Bb @ fb)


# Initialize Fourier solution on GLL pts
uf =  np.zeros(len(xb))
em =  np.zeros(int(N/2))
ef2 = np.zeros(int(N/2))
nN =  np.zeros(int(N/2))

for k in range(1, N + 1, 2):
    k1 = k + 1

    # ===== in the end of my derivation, u is independent of k ========
    # sk = np.sin(k * xb)
    # fk = 2 / (k * np.pi)
    # uk = 2 * fk / (1 + d2 * k * k)

    uf = 0.5*np.cos(xb)                     # solution of fourier
    ef = uf - us                            # error of fourier
    em[int(k1/2)-1]  = np.max(np.abs(ef))
    ef2[int(k1/2)-1] = np.sqrt(ef @ Bb @ ef / np.pi)
    nN[int(k1/2)-1]  = k
    if k == 1:
        uf1 = uf
    if k == 3:
        uf3 = uf
    if k == 5:
        uf5 = uf

fig, ax = plt.subplots()
ax.plot(xb, us, 'k-', linewidth=1.4, label = "Legendre")
ax.plot(xb, uf1, 'r--', label = "Fourier k=1")
ax.plot(xb, uf3, 'g--', label = "Fourier k=3")
ax.plot(xb, uf5, 'b--', label = "Fourier k=5")
ax.set_xlabel("x")
ax.set_ylabel("u(x)")
ax.legend()
plt.savefig('2_2_solution.pdf', bbox_inches="tight")
plt.show()

fig, ax = plt.subplots()
ax.plot(nN, ef2,          'b-',  linewidth=1.4, label = "Legendre")
ax.plot(nN, 0.3*nN**(-2), 'r--', label = r'$L_2$ error')
ax.plot(nN, em,           'g--', label = r'$L_\infty$ error')

ax.set_yscale('log')
ax.set_xscale('log')

ax.set_xlabel('N')
ax.set_ylabel(r'Error: $L_2 \ and \ L_\infty$')
ax.legend()
plt.savefig('2_2_error.pdf', bbox_inches="tight")
plt.show()

    \end{lstlisting}

\end{document}


